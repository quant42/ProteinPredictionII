\subsection{Dataset}

HypfuNN was developed at the rostlab\footnote{citation needed} along with other function prediction methods by other student groups. In order to better compare the fittness of the implemented prediction
methods, we all used the same dataset to identify similar proteins and transfer there annotations to the query sequence, as descriped in the Implementation part. The dataset contains 2815 HPO-annotated
proteins. [TODO: maybe say something about protein groups and/or possible data bias]

\subsection{Implementation}

As already described above, we predict protein annotations by homology. Proteins, that have the same function, so called homologs, tend to share a similar sequence. In order to identify proteins
with similar sequence, we are using blast and hhblits. For both, blast and hhblits, we require an minimal e-value of 1, to exclude weak similar protein sequences. Our tool supports to use
different blast e-value with the commandline option -e.

For each found similar sequences, we are then constructing a hpo tree.




For each query sequence our de novo predictor first constructs an hpo term tree from the 
