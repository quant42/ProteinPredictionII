\begin{abstract}

\section{Motivation:} 
Faced with a huge gap in the number of available sequences and available functional annotations, the prediction of protein function helps to identify research targets, understand diseases and close gaps in our knowledge of molecular processes. We use the available annotation data to transfer function descriptions to proteins with known sequence but unknown function (the standard case in public databases) from functionally characterized homologs. 
\section{Results:}
We identify homologs via blast and hhblits search in a database of annotated proteins and feed the annotations from these proteins to a neural network that assesses the confidence of a transfer and finetunes our prediction. To circumvent the difficulties of functional annotations in human language, we restrict annotations in training and prediction to terms from the human phenotype ontology (HPO). In a crossvalidation on a set of 2815 HPO-annotated proteins, we achieve an F-max measure of $0.xx \pm xx$. We also provide HPO annotations for the complete human proteome.

\section{Availability:} The datasets, predictor and predictions are available upon request.

\section{Contact:} \href{boidolj@in.tum.de}{boidolj@in.tum.de}, \href{spoeri@in.tum.de}{spoeri@in.tum.de}, \href{schoeffel@in.tum.de}{schoeffel@in.tum.de}
\end{abstract}