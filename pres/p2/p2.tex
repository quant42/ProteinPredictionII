\documentclass{beamer}

\usepackage[utf8]{inputenc}
%\usepackage{default}

\begin{document}

\begin{frame}
 \frametitle{ProteinPrediction II}
 PPII Ex3 - Week 3\\
 \hfill \\
 Jonathan Boidol, Rene Schoeffel, Yann Sp\"ori
\end{frame}

\begin{frame}[fragile]
 \frametitle{Mapping IDs}
 \begin{enumerate}
 \item \url{http://www.uniprot.org/mapping/} maps 2813 Uniprot ACs to 2801 Entrez Gene IDs
 \item python script \texttt{map.py}
         \begin{itemize}
         \item reads mapping and annotation files
         \item maps Uniprot AC to HPO term {\color{gray}\#or root node, but we have better annotations for everything}
         \end{itemize}
 \item output:\\
 \begin{verbatim}
         P00441 HP:0003394,HP:0002314,HP:0003202,HP:0010535
         P31749 HP:0000400,HP:0004322,HP:0004325
         P31213 HP:0000028,HP:0008736
         ...
 \end{verbatim}
\end{enumerate}
%277 annotations for P21802; 226 non-redundant annotations for P21802
\end{frame}

\begin{frame}[fragile]
 \frametitle{specific annotations}

        \begin{itemize}
        \item annotations are redundant (node and parents of annotation tree)
        \item use function \texttt{has\_children} to prune annotations
\begin{verbatim}
        P00441 HP:0003394,HP:0002314
        P31749 HP:0000400,HP:0004325
        P31213 HP:0008736
        ...
 \end{verbatim}
        \end{itemize}

 
\end{frame}

\begin{frame}[fragile]
 \frametitle{Graph data structure}

\begin{itemize}
	\item Graph object and node (= hpo term) object
	\item Graph can (only) be instanced by hpo file
	\item Graph is represented by a python dictionary
	\item SubGraphs by hpo ids may be created
		\begin{verbatim}
			Note, that at subGraphs don't change childnodes
		\end{verbatim}
	\item Functions (until now)
		\begin{itemize}
			\item +: get a subgraph which contains the nodoes from both graphs
			\item -: get a subgraph that contains only the nodes that had been in both graphs
			\item in: str: item with id, else node in graph
			\item getHpoTermById: get a term object by an id (None if not in subgraph, although it might be in graph)
			\item getHpoSubGraph: initialize a subgraph by leaves
			\item getLeaves: get all leaves of the subgraph
			\item getChildrens: get the children of a node
		\end{itemize}
\end{itemize}

\end{frame}

\end{document}
